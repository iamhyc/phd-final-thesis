% !TeX root = ../../main.tex

This thesis investigates the dynamic programming algorithm design for various emerging applications of multi-agent edge computing and edge learning systems.
The common challenge in these applications is how to handle the uncertainties in online decision making.
The uncertainties are raised by the randomness of job arrivals, transmission latencies, mobility of mobile devices, unknown wireless interference and etc.
% Moreover, there are many works focusing on the centralized algorithm design for edge computing systems by assuming a central operator with the full knowledge of the system, while the centralized dynamic programming algorithm design is not practical in many edge computing and edge learning scenarios:
% 1) the full knowledge of the system is hard to obtain due to significant signaling overhead particularly in large-scale systems;
% 2) the exchanged information is usually outdated due to the random network traffics and random mobility of mobile devices.
In this thesis, we would like to focus on two specific scenarios of edge computing and edge learning, and shed some light on the decentralized and approximate algorithm design.

In the first part of this thesis, we firstly introduce a centralized cooperative jobs dispatching problem in edge computing systems with multiple access points (APs) and edge servers.
Due to the uncertain traffic in the network between APs and edge servers, the job uploading delay can not be predicted accurately.
Since each job dispatching decision will affect the system state in the future, we formulate the joint optimization of jobs dispatching at all the APs as an infinite-horizon Markov decision process (MDP).
A novel approximate MDP solution framework is proposed via one-step policy iteration over a baseline policy to suppress the curse of dimensionality, where the analytical performance bound can be obtained.
Furthermore, the decentralized job dispatching problem with random network transmission latency is considered.
Specifically, the distributed dispatcher on each AP would receive partially and outdated information exchanged via periodic broadcast.
Hence, we formulate the distributed job dispatching problem by leveraging partially observable Markov decision process (POMDP) and propose a novel approximate MDP solution framework, called {\Dalgname}, to avoid the huge time complexity of conventional POMDP solutions.
Furthermore, we extend {\Dalgname} to handle a more general scenario where the priori knowledge of the system is absent.

\revise{
In Part II, we further explore the approximate algorithm design to edge learning scenarios.
Specifically, we consider both ``network-for-learning'' and ``learning-for-network'' scenarios in multi-agent edge network systems.
% Specifically, a group of {\IAVFullnames} ({\IAVs}) are moving along their planned routes to collect sensing data and train an object detection model via federated learning.
Firstly, we choose edge federated learning as a typical scenario for ``network-for-learning'', where the uplink model transmission is well scheduled in a vehicular edge federated learning system to achieve both time and energy efficiency.
We organize every $N$ time slots in the system as a super slot.
At the beginning of each super slot, the scheduling period and the global time allocation actions are joint optimized via solving a large-scale convex program.
Hence, the remaining rate allocation actions is decoupled for each vehicles and the local MDP problems is formulated for each vehicles.
We further propose a low-complexity algorithm inside each super slot for the local optimization problems.
Furthermore, we choose indoor Wi-Fi as a typical scenario for ``learning-for-network'', where a learning-based scheduling framework is proposed and implemented to optimize the application-layer quality-of-service (QoS) with cross-layer optimization.
Specifically, the contention window sizes are adjusted in media access control (MAC) layer, and the throughput of delay-sensitive tasks are throttled in application layer.
% Due to the unknown interference and vendor-specific implementation of wireless NIC, the relation between the scheduling policy and the system QoS is unknown.
The experiment results on the commercial-off-the-shelf testbed show that the proposed framework can achieve a significantly better QoS than the conventional EDCA mechanism and other heuristic algorithms.
}%
